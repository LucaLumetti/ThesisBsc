\section{Integrity}
Integrity assures to the receiver that a message is not corrupted or that an adversary modifies and relays it (for example in a man-in-the-middle attack).\\
The decryption of the message is not always needed to modify it but can be enough to have the ciphertext.\\
Hash functions are used to assure the integrity of a message.

\subsection{Hash Functions}
In general, hash functions are just functions that take arbitrary-length strings and compress them into shorter strings.
A hash function is a pair $(\mathsf{Gen}, \mathsf{H})$ such that:
\begin{itemize}
    \item{$\mathsf{Gen}$: is a randomized algorithm that takes as input a security parameter $n$ and outputs a key $s$}
    \item{$\mathsf{H}$: is a deterministic polynomial-time algorithm that takes as input a string $x \in \{0,1\}^*$ and a key $s$ to outputs a string $\mathsf{H}^s(x) \in \{0,1\}^{\mathit{l}(n)}$ where $\mathit{l}$ is a polynomial.}
\end{itemize}
In practice, hash functions are unkeyed, or rather it is included in the function itself.\\
As an example of use of hash functions, imagine that a user A want to send a message $m$ to an user B and he also want to assure its integrity. After they both agree on the hash function to use, A send $(m, \mathsf{H}(m))$, then upon receiving the pair, B itself calculate $\mathsf{H}(m)$ and verify that it is the same it received from A. If they match, $m$ can be considered intact.\\
The domain of $\mathsf{H}$ is unlimited, instead, its image is limited. For the pigeon-hole principles this means that there are infinite pairs of differents string $x$ and $x'$ such that $\mathsf{H}(x) = \mathsf{H}(x')$, this is also known as a \emph{collision}.

\subsection{Collision-resistant hash functions}
Hash functions used in cryptography are also called collision-resistant hash function, to emphatize the importance to have the property that no polynomial-time adversary can find them in a reasonable time.
There are 3 levels of security:
\begin{enumerate}
    \item{\textbf{Collision resistance:} is the most secure level and implies that, given the key $s$, is infeasible to find two different values $x$ and $x'$ such that $\mathsf{H}^s(x) = \mathsf{H}^s(x')$.}
    \item{\textbf{Second preimage resistance:} implies that, given $s$ and a string $x$, is infeasible for a polynomial-time algorithm to find a string $x'$ such that it collide with $x$}
    \item{\textbf{Preimage resistance:} implies that given the key $s$ and an hash $y$, is infeasible for a polynomial-time algorithm to find a value $x$ such that $\mathsf{H}^s(x) = y$.}
\end{enumerate}
Notice that every hash function that is collision resistant is second preimage resistant, also a second preimage resistant function is a preimage resistant function.\\

\subsection{Birthday Attack}
The probability of finding a collision by guessing two random string is inversely proportional to the number of bit outputted by the hash function. For example, SHA-256 has a 256-bit hash output resulting in $2^{256}$ possible results.\\
Now assume we have are given an hash function $\mathsf{H}^s : {0,1}^{*} \rightarrow {0,1}^{l(n)}$. Then we choose at random $q$ distinct strins $x_1,...,x_q$ and compute $y_i := \mathsf{H}^s(x_i)$, $\forall i \in {1,...,q}$. If $q \leq 2^{l(n)}$ the probability is:
$$
    1 - \prod_{i = 1}^{q-1} (1 - \frac{i}{2^{l(n)}})
$$
This problem has been extensively studied and is related to the so-colled birthday problem and an approximation of the above formula is:
$$
    \frac{q^2}{2\cdot2^{l(n)}}
$$
If we want to know how much $q$ are needed to have a probability at least of $50\%$ :
$$
    \frac{q^2}{2\cdot2^{ln(n)}} = 0.5
    \quad \implies \quad
    q = \sqrt{2^{ln(n)}}
$$
This mean that if the output length of a hash function is $l(n)$ bits then the birthday attack finds a collision in $\mathcal{O}(q) = \mathcal{O}(2^{l(n)})$ time.\\
