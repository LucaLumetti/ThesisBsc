With \emph{private-key cryptography} we refer to schemes that use a single key to encrypt and decrypt a message, for this reason, we will refer to them as \emph{symmetric} schemes.
A \textbf{private-key scheme} is a tuple $(\mathsf{Gen}, \mathsf{Enc}, \mathsf{Dec})$ such that:
\begin{itemize}
    \item{\textbf{$\mathsf{Gen}(\cdot)$:} is a randomized polynomial algorithm that generates the key. It takes as input a security parameter $n$ and outputs a key $k$ that satisfies $|k| \geq n$.\\
        We will write this as $k \leftarrow \mathsf{Gen}(1^n)$.}
    \item{\textbf{$\mathsf{Enc}(\cdot)$:} is a probabilistic polynomial-time algorithm that encrypts the message (or other forms of information) to send. It takes as input a key $k$ and a message $m$ to outputs a ciphertext $c$. We will refer to the unencrypted message also as plaintext.\\
        We will write this as $c \leftarrow \mathsf{Enc}_k(m)$.}
    \item{\textbf{$\mathsf{Dec}(\cdot)$:} is a deterministic polynomial-time algorith that takes as input a ciphertext $c$ and a key $k$, and outputs a plaintext $m$.\\
        We will write this as $m := \mathsf{Dec}_k(c)$}
\end{itemize}
It's also required that for every $n$, every $k$ and every $m$ it holds that $m = \mathsf{Dec}_k(\mathsf{Enc}_k(m))$.\\
Now we want to look which tools are used in the contruction of secure private-key scheme.

\section{Pseudorandom Permutations}
Pseudorandom functions are function that map n-bits strings to n-bit strings and that cannot be distinguished from a random permutation chosen uniformely from every function that map n-bit string to n-bit string.
The first set of function, for a key of length $s$ bit, have a cardinality of $2^{s}$ while the second set have a cardinality of $2^{n\cdot2^n}$.\\
Pseudorandom permutations are pseudorandom functions with some extra proprieties:
Let $F : \{0,1\}^{n} \times \{0,1\}^{s} \rightarrow \{0,1\}^{n}$ be an efficient, length-preserving, keyed function and $F_k(m) := F(m, k)$.
$F$ is a pseudorandom permutation (PRP) if:
\begin{itemize}
    \item{$\forall k \in \{0,1\}^{s}$, $F$ is a bijection from $\{0,1\}^{n}$ to $\{0,1\}^{n}$}.
    \item{$\forall k \in \{0,1\}^{s}$ exists an efficient algorith $F^{-1}_k$.}
    \item{For all probabilistic polynomial-time distinguishers $D$:
        $$
            |\mathsf{Pr}[D^{F_k}(n) = 1] - \mathsf{Pr}[D^{f_n}(n) = 1]| < \mathsf{negl}(n)
        $$
        where $k$ is chosen uniformly at random from $\{0,1\}^{s}$ and $f_n$ is chosen uniformly at random from the set of every permutations on n-bit strings.
        }
\end{itemize}

\section{Block Chipers}
Block cipher are PRPs families that operate one a block of a fixed length. To ensure security against CPA, there are varius mode of operation for block ciphers, like \emph{Cipher Block Chaining} (CBC) and \emph{Counter Mode} (CTR).
