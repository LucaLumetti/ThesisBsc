\documentclass[12pt,a4paper,oneside]{book}
\usepackage[utf8]{inputenc}
\usepackage[T1]{fontenc}
\usepackage[italian]{babel}
\usepackage [a4paper,left=3cm,bottom=3.5cm,right=3cm,top=3cm]{geometry}
\usepackage{cite}
\usepackage{mathtools}
\renewcommand{\baselinestretch}{1.2}
\usepackage[]{frontespizio}

\begin{document}

\begin{frontespizio}
\Rientro {1.5cm}
\Margini {1.5cm}{1.5cm}{1.5cm}{1.0cm}
\Universita {Modena e Reggio Emilia}
\Logo [1.5cm]{img/UNIMORE}
\Facolta {Ingegneria}
\Corso [Laurea]{Ingegneria Informatica}
\Annoaccademico {2019-2020}
\Titoletto {Prova Finale}
\Titolo {Non so ancora il titolo}
\Candidato [118447]{Luca Lumetti}
\Relatore {Prof. Maurizio Casoni}
\Correlatore{Dott. Martin Klapez}
\end{frontespizio}

\newpage

\tableofcontents
\listoffigures
\listoftables


% struttura (?)
% Schema crittografico
% - simmetrico
% - asimmetrico
% - confidenzialità
% - integrità
% - autenticazione
% - segretezza perfetta
% - segretezza computazionale.
% - attacchi? <--- sotto confidenzialità
% - schema sicuro sotto cca? <-- sotto confidenzialità
% - block chiper? <-- sotto confidenzialità

\chapter{Introduzione}\label{c:introduzione}
Introduzione


\chapter{Schemi a chiave privata}
Gli schemi crittografici a chiave privata (anche detti simmetrici) sono stati i primi schemi ad essere creati e sono tutt'ora molto usati. Sono caratterizzati dalla presenza di una sola chiave che viene utilizzata sia per criptare un messaggio sia per decriptarlo.\\
Più precisamente, uno schema crittografico a chiave privata è una tupla $(\mathsf{Gen}, \mathsf{Enc}, \mathsf{Dec})$ dove:
\begin{itemize}
    \item{\textbf{$\mathsf{Gen}(\cdot)$:} è un algoritmo randomizzato che a fronte di un input $1^{n}$ genera una chiave $k$ tale che $|k| \geq n$. E' quindi definito nel seguente modo: $k \leftarrow \mathsf{Gen}(1^{n})$}. Il parametro $n$ viene detto \emph{parametro di sicurezza}
    \item{\textbf{$\mathsf{Enc}(\cdot)$:} è un algoritmo polinomiale che puo' essere di tipo deterministico o probabilistico. Ha come input una chiave $k$ e un messaggio $m$ e restituisce un messaggio cifrato $c$. Potendo essere un algoritmo probabilistico, scriveremo: $c \leftarrow \mathsf{Enc}_k(m)$.}
    \item{\textbf{$\mathsf{Dec}(\cdot)$:} è un algoritmo polinomiale deterministico che a ha come input una chiave $k$ e un messaggio cifrato $c$ e restituisce un messaggio $m$. Essendo un algoritmo deterministico scriveremo: $m := \mathsf{Dec}_k(c)$.}
\end{itemize}
Si definisce inoltre con $\mathcal{M}$ lo spazio di tutti i messaggi che possono essere inviati $\mathcal{K}$ lo spazio di tutte le chiavi. Questi due insiemi variano dipendentemente da come vengono implementate 3 funzioni dello schema.\\
Inoltre, per ogni $n$, per ogni $k \in \mathcal{K}$ e per ogni $m \in \mathcal{M}$ deve valere la seguente relazione:
$$
    m = \mathsf{Dec}_k(\mathsf{Enc}_k(m))
$$


\section{Confidenzialità}
La confidenzialità assicura che in una comunicazione, un \emph{eavesdropper}, ovvero una persona esterna che intercetta i messaggi scambiati tra le due parti di una comunicazione, non possa ottenere alcune informazione utile. In altre parole significa che il testo cifrato non fa trasparire nessuna informazione riguardante il messaggio originale o la chiave.\\
\subsection{Perfectly Secret}
Dalla definizione di confidenzialità deriva direttamente quella di schema crittografico "perfectly secret". Riprendendo la notazione utilizzata per definire uno schema crittografico a chiave privata, definiamo con $\mathsf{Pr}[M = m]$ la probabilità che il messaggio $m$ venga trasmesso e con $\mathsf{Pr}[C = c]$ la probabilità che $\mathsf{Enc}_k(m)$ sia $c$. Allora uno schema è "perfectly secret" se:
$$
    \mathsf{Pr}[M = m | C = c] = \mathsf{Pr}[M = m]
$$
Ovvero la distribuzione dei messaggi in chiaro è indipendente dalla distrubuzione dei messaggi cifrati.\\
Questa definizione però ha delle conseguenze sulla cardinalità dello spazio dei messaggi $\mathcal{M}$ e delle chiavi $\mathcal{K}$ in quanto $|\mathcal{K}| \geq |\mathcal{M}|$ è condizione necessaria affinchè uno schema sia "perfectly secret". per questo motivo nella pratica non è una caratteristica molto usata.

\subsection{Computationally Secret}
Gli schemi di crittografia moderni e attualmente utilizzati non soddisfano la definizione di "perfectly secret", ovvero possono essere violati se si dispone di sufficente tempo e potenza computazionale, ma la quantità di tempo necessaria è dell'ordine di qualche decina di anni o più anche utilizzando i più grandi supercomputer oggi esistenti, in questo caso si parla quindi di schema "computationally secure".
Questa definizione include due rilassamenti rispetto a quella di "perfectly secret":
\begin{itemize}
    \item{L'avversario è \emph{efficiente}, ovvero un avversario che utilizza un algoritmo probabilistico polinomiale rispetto al valore $n$ dello schema}
    \item{L'avversario ha sempre una probabilità di successo, ma sufficentemente bassa da considerarla trascurabile in n}
\end{itemize}
Una funzione $f$ viene considerata trascurabile (negligible) se per ogni polinomio $p(\cdot)$, esiste un valore $N$ per cui:
$$
    f(n) < \frac{1}{p(n)}
$$
E si incherà con: $f(n) = \mathsf{negl}(n)$. \\
Introduciamo ora un esperimento eseguito su uno schema $\Pi$ a chiave privata (PrivK) in presenza di un avversario intercettatore (eavesdropper) $\mathcal{A}$ di tipo probabilistico polinomiale. L'esperimento viene eseguito su due messaggi $m_0$ e $m_1$ e solo uno dei due, scelto casualmente, viene cifrato. \\
Sia $b$ un numero casuale tra 0 e 1, si ha: $c \leftarrow \mathsf{Enc}_k(m_b)$.\\
L'avversario $\mathcal{A}$ conosce $m_0$, $m_1$ e $c$.
L'esperimento, che chiameremo $\mathsf{PrivK}^{\mathsf{eav}}_{\mathcal{A}, \Pi}$, ha valore $1$ se $\mathcal{A}$ indovina il valore di $b$, altrimenti $0$ \\
Grazie a questo esperimento, è possibile definire uno schema "computationally secret" in presenza di un intercettatore, nel seguente modo:
$$
   \mathsf{Pr}[\mathsf{PrivK}^{\mathsf{eav}}_{\mathcal{A}, \Pi}(n) = 1] \leq \frac{1}{2} + \mathsf{negl}(n)
$$
E' importante notare che $\mathcal{A}$ conosce oltre a $c$ anche $m_0$, $m_1$, ma comunque non è in grado di determinare quale dei due messaggi sia stato codificato.

\section{Integrità}
L'integrità assicura che un messaggio non venga modificato da una terza parte che si inserisce nella comunicazione. E' importante notare che in alcuni casi, un messaggio può essere modificato anche senza essere decodificato, ad esempio togliendo parti del ciphertext, scambiandone l'ordine oppure aggiungendo parti copiandone alcune dallo stesso ciphertext.
Per assicurare l'integrità di uno schema a chiave privata si fa uso delle funzioni di \emph{Hash}
\subsection{Funzioni di Hash}
Le funzioni di Hash sono funzioni che \emph{comprimono} un messaggio di lunghezza arbitraria in una stringa generalmente più corta.\\
Più precisamente, una funzione di hash è una coppia $(\mathsf{Gen}, \mathsf{H})$ tale che:
\begin{itemize}
    \item{$\mathsf{Gen}$ è un algoritmo probabilistico e polinomiale che a fronte di un input $1^n$ restituisce come output una chiave $s$}
    \item{$\mathsf{H}$ è un algoritmo polinomiale deterministico con input una stringa $x \in \{0, 1\}^{*}$  e output una stringa $\mathsf{H}^{s}(x) \in \{0, 1\}^{l(n)}$ con $l$ polinomio in $n$.}
\end{itemize}
Dalla definizione emerge immediatamente un problema comune a tutte le funzioni di hash: l'esistenza di collisioni. Una collisione avviene quando due stringhe $x_1$ e $x_2$ diverse tra loro hanno lo stesso hash $\mathsf{H}^{s}(x_1) = \mathsf{H}^{s}(x_2)$. Essendo lo spazio di input illimitato a differenza di quello di output che è limitato, è ovvio che il numero di collisioni sia a sua volta infinito. Per questo motivo, un'importante caratteristica che le funzioni di hash devono avere è quella della resistenza a collisioni, ovvero rendere difficile il trovarle.\\
Nella pratica le funzioni di hash non utilizzano una chiave e quindi vengono definite solamente delle funzioni di hash
In questo caso, volendo solo assicurare l'integrità del messaggio e nient'altro, il mittente del messaggio può inviare la coppia $(m, \mathsf{H}(m))$. In questo modo, il ricevente può verificare l'integrità del messaggio calcolando a sua volta $\mathsf{H}(m)$ e confrontando il risultato con quello ricevuto. Se i due hash corrispondono, allora il messaggio è considerabile valido.\\
\subsection{Birthday Attack}


\input{chpt/chiave-privata/autenticazione.tex}

%\chapter{Introduzione}\label{c:introduzione}
Introduzione


\section{Confidenzialità}
La confidenzialità assicura che in una comunicazione, un \emph{eavesdropper}, ovvero una persona esterna che intercetta i messaggi scambiati tra le due parti di una comunicazione, non possa ottenere alcune informazione utile. In altre parole significa che il testo cifrato non fa trasparire nessuna informazione riguardante il messaggio originale o la chiave.\\
\subsection{Perfectly Secret}
Dalla definizione di confidenzialità deriva direttamente quella di schema crittografico "perfectly secret". Riprendendo la notazione utilizzata per definire uno schema crittografico a chiave privata, definiamo con $\mathsf{Pr}[M = m]$ la probabilità che il messaggio $m$ venga trasmesso e con $\mathsf{Pr}[C = c]$ la probabilità che $\mathsf{Enc}_k(m)$ sia $c$. Allora uno schema è "perfectly secret" se:
$$
    \mathsf{Pr}[M = m | C = c] = \mathsf{Pr}[M = m]
$$
Ovvero la distribuzione dei messaggi in chiaro è indipendente dalla distrubuzione dei messaggi cifrati.\\
Questa definizione però ha delle conseguenze sulla cardinalità dello spazio dei messaggi $\mathcal{M}$ e delle chiavi $\mathcal{K}$ in quanto $|\mathcal{K}| \geq |\mathcal{M}|$ è condizione necessaria affinchè uno schema sia "perfectly secret". per questo motivo nella pratica non è una caratteristica molto usata.

\subsection{Computationally Secret}
Gli schemi di crittografia moderni e attualmente utilizzati non soddisfano la definizione di "perfectly secret", ovvero possono essere violati se si dispone di sufficente tempo e potenza computazionale, ma la quantità di tempo necessaria è dell'ordine di qualche decina di anni o più anche utilizzando i più grandi supercomputer oggi esistenti, in questo caso si parla quindi di schema "computationally secure".
Questa definizione include due rilassamenti rispetto a quella di "perfectly secret":
\begin{itemize}
    \item{L'avversario è \emph{efficiente}, ovvero un avversario che utilizza un algoritmo probabilistico polinomiale rispetto al valore $n$ dello schema}
    \item{L'avversario ha sempre una probabilità di successo, ma sufficentemente bassa da considerarla trascurabile in n}
\end{itemize}
Una funzione $f$ viene considerata trascurabile (negligible) se per ogni polinomio $p(\cdot)$, esiste un valore $N$ per cui:
$$
    f(n) < \frac{1}{p(n)}
$$
E si incherà con: $f(n) = \mathsf{negl}(n)$. \\
Introduciamo ora un esperimento eseguito su uno schema $\Pi$ a chiave privata (PrivK) in presenza di un avversario intercettatore (eavesdropper) $\mathcal{A}$ di tipo probabilistico polinomiale. L'esperimento viene eseguito su due messaggi $m_0$ e $m_1$ e solo uno dei due, scelto casualmente, viene cifrato. \\
Sia $b$ un numero casuale tra 0 e 1, si ha: $c \leftarrow \mathsf{Enc}_k(m_b)$.\\
L'avversario $\mathcal{A}$ conosce $m_0$, $m_1$ e $c$.
L'esperimento, che chiameremo $\mathsf{PrivK}^{\mathsf{eav}}_{\mathcal{A}, \Pi}$, ha valore $1$ se $\mathcal{A}$ indovina il valore di $b$, altrimenti $0$ \\
Grazie a questo esperimento, è possibile definire uno schema "computationally secret" in presenza di un intercettatore, nel seguente modo:
$$
   \mathsf{Pr}[\mathsf{PrivK}^{\mathsf{eav}}_{\mathcal{A}, \Pi}(n) = 1] \leq \frac{1}{2} + \mathsf{negl}(n)
$$
E' importante notare che $\mathcal{A}$ conosce oltre a $c$ anche $m_0$, $m_1$, ma comunque non è in grado di determinare quale dei due messaggi sia stato codificato.

\section{Integrità}
L'integrità assicura che un messaggio non venga modificato da una terza parte che si inserisce nella comunicazione. E' importante notare che in alcuni casi, un messaggio può essere modificato anche senza essere decodificato, ad esempio togliendo parti del ciphertext, scambiandone l'ordine oppure aggiungendo parti copiandone alcune dallo stesso ciphertext.
Per assicurare l'integrità di uno schema a chiave privata si fa uso delle \emph{funzioni di hash}

\subsection{Funzioni di Hash}
Le funzioni di Hash sono funzioni che \emph{comprimono} un messaggio di lunghezza arbitraria in una stringa generalmente più corta.\\
Più precisamente, una funzione di hash è una coppia $(\mathsf{Gen}, \mathsf{H})$ tale che:
\begin{itemize}
    \item{$\mathsf{Gen}$ è un algoritmo probabilistico e polinomiale che a fronte di un input $1^n$ restituisce come output una chiave $s$}
    \item{$\mathsf{H}$ è un algoritmo polinomiale deterministico con input una stringa $x \in \{0, 1\}^{*}$  e output una stringa $\mathsf{H}^{s}(x) \in \{0, 1\}^{l(n)}$ con $l$ polinomio in $n$.}
\end{itemize}
Inoltre una funzione di hash deve avere le seguenti proprietà:
\begin{enumerate}
    \item{Non deve essere invertibile, deve quindi essere impossibile ricavare la stringa $x$ partendo dal suo hash $\mathsf{H}^{s}(s)$.}
    \item{Deve essere impossibile trovare un messaggio $x$ tale che il suo hash sia un valore dato.}
    \item{Anche solo un piccolo cambiamento nella stringa $x$ deve modificare in modo considerevole il nuovo hash in modo che appaia incorrelato con quello della stringa originale (effetto valanga).}
    \item{Deve essere impossibile trovare due messaggi diversi che hanno lo stesso hash (resistente alle collisioni).}
\end{enumerate}
Nella pratica le funzioni di hash non utilizzano una chiave (più precisamente la chiave è inclusa nella funzione) e quindi vengono definite solamente delle funzioni di hash, alcune delle più note sono MD5, SHA-1, SHA-256.
Nella crittografia le funzioni di hash vengono utilizzate all'interno delle funzioni di MAC che vedremo nel successivo capitolo relativo all'autenticazione.
Vengono comunque utilizzate in svariati campi sempre per assicurare l'integrità dei dati ma sono utilizzate anche nelle hash table per salvare e accedere ad un dato in un tempo praticamente costante.
Per dare comunque un'idea di come una funzione di hash possa assicurare l'integrità si può pensare al seguente caso:\\
Assumiamo che due parti, A e B, devono comunicare tra loro. In particolare A deve mandare un messaggio $m$ a B. Il mittente invia la coppia $(m, \mathsf{H}(m))$, in questo modo, il ricevente può verificare l'integrità del messaggio calcolando a sua volta $\mathsf{H}(m)$ e confrontando il risultato con quello ricevuto. Se i due hash corrispondono, allora il messaggio è considerabile valido.\\

\subsection{Collisioni} % forse potrei aggiungergi il birthday attack
Riprendendo la definizione di funzione di hash, si può notare che essendo il dominio della funzione illimitato a differenza del codominio, esisteranno un numero infinito di stringhe il cui hash è identico (principio del cassetto).
In questo caso, quando due stringhe $x_1 \neq x_2 \rightarrow \mathsf{H}(x_1) = \mathsf{H}(x_2)$ si dice  che si ha una \emph{collisione}.
Quando si fa riferimento alle funzioni di hash utilizzare nell'ambito della crittografia, si parla di 3 requisiti di sicurezza:
\begin{enumerate}
    \item{\textbf{Collision resistance:} è il requisito più difficile da ottenere ed implica che, per una funzione di hash, data la chiave $s$, per un algoritmo polinomiale sia difficile trovare due valori $x_1 \neq x_2$ per cui $\mathsf{H}^s(x_1) = \mathsf{H}^s(x_2)$.}
    \item{\textbf{Second preimage resistance:} implica che data la chiave $s$ e una stringa $x_1$, sia difficile per un algoritmo polinomiale trovare una stringa $x_2$ per cui $\mathsf{H}^s(x_1) = \mathsf{H}^s(x_2)$}.
    \item{\textbf{Preimage resistance:} implica che, data una chiave $s$ e un hash $y$, sia difficile per un algoritmo polinomiale trovare un valore $x$ per cui $\mathsf{H}^s(x) = y$.}
\end{enumerate}
Si nota che $1 \implies 2$ e $2 \implies 3$.

\input{chiave-privata/autenticazione.tex}

\chapter{Conclusione}\label{c:conclusione}
Conclusione


\appendix

\bibliographystyle{plain}
\bibliography{thesis}

\end{document}
