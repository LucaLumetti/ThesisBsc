\section{Integrità}
L'integrità assicura che un messaggio non venga modificato da una terza parte che si inserisce nella comunicazione. E' importante notare che in alcuni casi, un messaggio può essere modificato anche senza essere decodificato, ad esempio togliendo parti del ciphertext, scambiandone l'ordine oppure aggiungendo parti copiandone alcune dallo stesso ciphertext.
Per assicurare l'integrità di uno schema a chiave privata si fa uso delle funzioni di \emph{Hash}
\subsection{Funzioni di Hash}
Le funzioni di Hash sono funzioni che \emph{comprimono} un messaggio di lunghezza arbitraria in una stringa generalmente più corta.\\
Più precisamente, una funzione di hash è una coppia $(\mathsf{Gen}, \mathsf{H})$ tale che:
\begin{itemize}
    \item{$\mathsf{Gen}$ è un algoritmo probabilistico e polinomiale che a fronte di un input $1^n$ restituisce come output una chiave $s$}
    \item{$\mathsf{H}$ è un algoritmo polinomiale deterministico con input una stringa $x \in \{0, 1\}^{*}$  e output una stringa $\mathsf{H}^{s}(x) \in \{0, 1\}^{l(n)}$ con $l$ polinomio in $n$.}
\end{itemize}
Dalla definizione emerge immediatamente un problema comune a tutte le funzioni di hash: l'esistenza di collisioni. Una collisione avviene quando due stringhe $x_1$ e $x_2$ diverse tra loro hanno lo stesso hash $\mathsf{H}^{s}(x_1) = \mathsf{H}^{s}(x_2)$. Essendo lo spazio di input illimitato a differenza di quello di output che è limitato, è ovvio che il numero di collisioni sia a sua volta infinito. Per questo motivo, un'importante caratteristica che le funzioni di hash devono avere è quella della resistenza a collisioni, ovvero rendere difficile il trovarle.\\
Nella pratica le funzioni di hash non utilizzano una chiave e quindi vengono definite solamente delle funzioni di hash
In questo caso, volendo solo assicurare l'integrità del messaggio e nient'altro, il mittente del messaggio può inviare la coppia $(m, \mathsf{H}(m))$. In questo modo, il ricevente può verificare l'integrità del messaggio calcolando a sua volta $\mathsf{H}(m)$ e confrontando il risultato con quello ricevuto. Se i due hash corrispondono, allora il messaggio è considerabile valido.\\
\subsection{Birthday Attack}

