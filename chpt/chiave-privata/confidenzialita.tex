\section{Confidenzialità}
La confidenzialità assicura che in una comunicazione, un \emph{eavesdropper}, ovvero una persona esterna che intercetta i messaggi scambiati tra le due parti di una comunicazione, non possa ottenere alcune informazione utile. In altre parole significa che il testo cifrato non fa trasparire nessuna informazione riguardante il messaggio originale o la chiave.\\
\subsection{Perfectly Secret}
Dalla definizione di confidenzialità deriva direttamente quella di schema crittografico "perfectly secret". Riprendendo la notazione utilizzata per definire uno schema crittografico a chiave privata, definiamo con $\mathsf{Pr}[M = m]$ la probabilità che il messaggio $m$ venga trasmesso e con $\mathsf{Pr}[C = c]$ la probabilità che $\mathsf{Enc}_k(m)$ sia $c$. Allora uno schema è "perfectly secret" se:
$$
    \mathsf{Pr}[M = m | C = c] = \mathsf{Pr}[M = m]
$$
Ovvero la distribuzione dei messaggi in chiaro è indipendente dalla distrubuzione dei messaggi cifrati.\\
Questa definizione però ha delle conseguenze sulla cardinalità dello spazio dei messaggi $\mathcal{M}$ e delle chiavi $\mathcal{K}$ in quanto $|\mathcal{K}| \geq |\mathcal{M}|$ è condizione necessaria affinchè uno schema sia "perfectly secret". per questo motivo nella pratica non è una caratteristica molto usata.

\subsection{Computationally Secret}
Gli schemi di crittografia moderni e attualmente utilizzati non soddisfano la definizione di "perfectly secret", ovvero possono essere violati se si dispone di sufficente tempo e potenza computazionale, ma la quantità di tempo necessaria è dell'ordine di qualche decina di anni o più anche utilizzando i più grandi supercomputer oggi esistenti, in questo caso si parla quindi di schema "computationally secure".
Questa definizione include due rilassamenti rispetto a quella di "perfectly secret":
\begin{itemize}
    \item{L'avversario è \emph{efficiente}, ovvero un avversario che utilizza un algoritmo probabilistico polinomiale rispetto al valore $n$ dello schema}
    \item{L'avversario ha sempre una probabilità di successo, ma sufficentemente bassa da considerarla trascurabile in n}
\end{itemize}
Una funzione $f$ viene considerata trascurabile (negligible) se per ogni polinomio $p(\cdot)$, esiste un valore $N$ per cui:
$$
    f(n) < \frac{1}{p(n)}
$$
E si incherà con: $f(n) = \mathsf{negl}(n)$. \\
Introduciamo ora un esperimento eseguito su uno schema $\Pi$ a chiave privata (PrivK) in presenza di un avversario intercettatore (eavesdropper) $\mathcal{A}$ di tipo probabilistico polinomiale. L'esperimento viene eseguito su due messaggi $m_0$ e $m_1$ e solo uno dei due, scelto casualmente, viene cifrato. \\
Sia $b$ un numero casuale tra 0 e 1, si ha: $c \leftarrow \mathsf{Enc}_k(m_b)$.\\
L'avversario $\mathcal{A}$ conosce $m_0$, $m_1$ e $c$.
L'esperimento, che chiameremo $\mathsf{PrivK}^{\mathsf{eav}}_{\mathcal{A}, \Pi}$, ha valore $1$ se $\mathcal{A}$ indovina il valore di $b$, altrimenti $0$ \\
Grazie a questo esperimento, è possibile definire uno schema "computationally secret" in presenza di un intercettatore, nel seguente modo:
$$
   \mathsf{Pr}[\mathsf{PrivK}^{\mathsf{eav}}_{\mathcal{A}, \Pi}(n) = 1] \leq \frac{1}{2} + \mathsf{negl}(n)
$$
E' importante notare che $\mathcal{A}$ conosce oltre a $c$ anche $m_0$, $m_1$, ma comunque non è in grado di determinare quale dei due messaggi sia stato codificato.
